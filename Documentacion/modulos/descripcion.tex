\section{Descripción e Información del Proyecto:}

\subsection{Objetivos:}
NotaBoard es una aplicación diseñada para proporcionar a los usuarios una plataforma simple y efectiva para tomar y organizar notas. Los objetivos principales del proyecto incluyen:
\begin{itemize}
    \item Facilitar la toma de notas rápidas y fáciles de acceder.
    \item Proporcionar una interfaz intuitiva y fácil de usar para organizar y gestionar notas.
    \item Permitir la sincronización y el acceso a las notas desde múltiples dispositivos para una experiencia de usuario sin interrupciones.
    \item Mejorar la productividad y la eficiencia al tener todas las notas importantes en un solo lugar.
\end{itemize}

\subsection{Beneficios:}
\begin{itemize}
    \item Acceso rápido y fácil a las notas en cualquier momento y lugar.
    \item Organización eficiente de notas mediante la creación de tableros y categorías personalizadas.
    \item Sincronización automática de notas entre dispositivos para mantener la información actualizada.
    \item Mejora de la productividad al proporcionar una forma rápida y conveniente de capturar y gestionar información importante.
\end{itemize}

\subsection{Restricciones:}
\begin{itemize}
    \item El proyecto debe completarse dentro del marco de tiempo establecido y dentro del presupuesto asignado.
    \item La aplicación debe ser compatible con una variedad de dispositivos y sistemas operativos para garantizar una amplia accesibilidad.
    \item Se deben implementar medidas de seguridad robustas para proteger la privacidad y la confidencialidad de las notas de los usuarios.
\end{itemize}

\subsection{Stakeholders:}
\begin{enumerate}
    \item \textbf{Desarrolladores:} Responsables de diseñar, desarrollar y mantener la aplicación.
    \item \textbf{Usuarios Finales:} Personas que utilizan la aplicación para tomar y organizar sus notas.
    \item \textbf{Gerencia y Patrocinadores del Proyecto:} Responsables de proporcionar recursos y apoyo para el desarrollo del proyecto.
    \item \textbf{Equipo de Calidad:} Encargado de garantizar que la aplicación cumpla con los estándares de calidad y funcionalidad establecidos.
    \item \textbf{Equipo de Soporte:} Responsable de brindar asistencia técnica y resolver problemas relacionados con la aplicación.
\end{enumerate}