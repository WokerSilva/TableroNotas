\section*{Historia de Usuario:}

\subsection*{Nombre:} 
Sara, Estudiante Universitaria

\subsection*{Rol:} 
Estudiante

\subsection*{Necesidad/Objetivo:}
Sara necesita una manera fácil y rápida de organizar sus notas y apuntes para sus clases universitarias. Actualmente, está lidiando con múltiples cuadernos y hojas sueltas, lo que dificulta encontrar la información que necesita cuando la necesita. Quiere una solución digital que le permita organizar sus notas de manera eficiente y acceder a ellas desde cualquier lugar.

\subsection*{Acciones:}
\begin{enumerate}[label=\arabic*.]
    \item Sara descarga la aplicación NotaBoard desde la tienda de aplicaciones de su teléfono.
    \item Abre la aplicación y crea una cuenta.
    \item Explora la interfaz de la aplicación y encuentra la opción para crear un nuevo tablero de notas.
    \item Crea tableros separados para cada una de sus clases, como "Matemáticas", "Historia" y "Literatura".
    \item Dentro de cada tablero, agrega notas individuales para temas específicos, como "Álgebra", "Guerras Mundiales" y "Shakespeare".
    \item Organiza sus notas arrastrándolas y soltándolas en el orden deseado.
    \item Accede a sus notas desde su teléfono, tableta o computadora portátil mientras estudia en la biblioteca o en casa.
\end{enumerate}

\subsection*{Resultado:}
Sara ahora tiene todas sus notas organizadas de manera digital en un solo lugar, lo que facilita su acceso y estudio. Ya no tiene que preocuparse por perder apuntes o llevar consigo múltiples cuadernos. La aplicación NotaBoard le permite ser más eficiente y productiva en sus estudios universitarios.

\section*{Mapa de Empatía:}

\subsection*{¿Qué piensa y siente?}
\begin{itemize}
    \item Sara: Quiere sentirse organizada y preparada para sus clases. Está frustrada por la cantidad de papeles que tiene que llevar consigo y desea una solución digital para simplificar su vida académica.
\end{itemize}

\subsection*{¿Qué ve?}
\begin{itemize}
    \item Sara ve una montaña de papeles dispersos por su escritorio y mochila. Ve a sus compañeros de clase utilizando aplicaciones para tomar notas y organizar su trabajo de manera digital.
\end{itemize}

\subsection*{¿Qué escucha?}
\begin{itemize}
    \item Sara escucha a sus amigos hablar sobre las aplicaciones que utilizan para organizar sus notas y mejorar su productividad. También escucha a sus profesores mencionar la importancia de la organización en el éxito académico.
\end{itemize}

\subsection*{¿Qué dice y hace?}
\begin{itemize}
    \item Sara está investigando diferentes aplicaciones de notas en línea y pregunta a sus amigos cuál recomiendan. Está dispuesta a probar nuevas herramientas tecnológicas para mejorar su experiencia de estudio.
\end{itemize}

\subsection*{¿Cuáles son sus dolores?}
\begin{itemize}
    \item Sara se siente abrumada por la cantidad de papeles que tiene que manejar y la dificultad para encontrar la información que necesita cuando la necesita. Está preocupada por perder notas importantes y no poder recuperarlas a tiempo para sus exámenes.
\end{itemize}

\subsection*{¿Cuáles son sus alegrías?}
\begin{itemize}
    \item Sara se siente emocionada por la posibilidad de tener todas sus notas organizadas en un solo lugar y poder acceder a ellas fácilmente desde cualquier dispositivo. Está ansiosa por mejorar su productividad y eficiencia en sus estudios universitarios.
\end{itemize}