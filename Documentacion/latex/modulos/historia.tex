\section{Historia de Usuario:}

\subsection*{Nombre:} 
Sara, Estudiante Universitaria

\subsection*{Rol:} 
Estudiante

\subsection*{Necesidad/Objetivo:}
Sara necesita una manera fácil y rápida de organizar sus notas y apuntes para sus clases universitarias. Actualmente, está lidiando con múltiples cuadernos y hojas sueltas, lo que dificulta encontrar la información que necesita cuando la necesita. Quiere una solución digital que le permita organizar sus notas de manera eficiente y acceder a ellas desde cualquier lugar.

\subsection*{Acciones:}
\begin{enumerate}[label=\arabic*.]
    \item Sara descarga la aplicación NotaBoard desde la tienda de aplicaciones de su teléfono.
    \item Abre la aplicación y crea una cuenta.
    \item Explora la interfaz de la aplicación y encuentra la opción para crear un nuevo tablero de notas.
    \item Crea tableros separados para cada una de sus clases, como "Matemáticas", "Historia" y "Literatura".
    \item Dentro de cada tablero, agrega notas individuales para temas específicos, como "Álgebra", "Guerras Mundiales" y "Shakespeare".
    \item Organiza sus notas arrastrándolas y soltándolas en el orden deseado.
    \item Accede a sus notas desde su teléfono, tableta o computadora portátil mientras estudia en la biblioteca o en casa.
\end{enumerate}

\subsection*{Resultado:}
Sara ahora tiene todas sus notas organizadas de manera digital en un solo lugar, lo que facilita su acceso y estudio. Ya no tiene que preocuparse por perder apuntes o llevar consigo múltiples cuadernos. La aplicación NotaBoard le permite ser más eficiente y productiva en sus estudios universitarios.

